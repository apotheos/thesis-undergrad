\newpage \appendix \appendixpage \addappheadtotoc
\section{Indexed Book Statistics}
\label{appendix:d}

Book Book Goose~\cite{book-book-goose}, a random book browsing site, was used to generate a list of 100 random books.\footnote{By inspecting network activity while on the site, it is evident that Book Book Goose uses Amazon's product API\cite{amazon-products} to randomly pull book information from Amazon.}
Of these books, the books that did not have an author and a title were discarded, leaving 77 books.
Each of these books were searched in Google Books~\cite{google-books}.
If a book in the search results matched the book searched, the number of pages the book contained were recorded.
If the ``preview'' functionality was enabled for the book, the table of contents would be searched for the word ``Index'', and the results of that search (success or failure) were recorded.
Here are the results of this mini-study:

\begin{center}
\begin{tabular}{|l|r|}
\hline 
Number of books located in Google Books & 58 books \\ 
\hline 
Number of books with previews in Google Books & 32 books \\ 
\hline 
Number of books with preview and index & 15 books \\ 
\hline 
Percent of books with index & 47\% books \\ 
\hline
Average pages per book & 350 pages \\ 
\hline 
Average pages per indexed book & 380 pages \\ 
\hline 
\end{tabular}
\end{center}


\section{Database Schema Definition}
\label{appendix:a}
\lstinputlisting[language=SQL]{code/mysql/schemaDefinition.sql}

\section{\Naive Bayes Classifier (classifier.py)}
\label{appendix:b}
\lstinputlisting[language=Python]{code/tools/classifier.py}

\section{Other Data and Source Code}
\label{appendix:c}

Throughout the course of this paper, this research mentions several tools and data sets whose source does not appear above in order to keep the number of printed pages to a reasonable number.
All of theses tools, and the source of this paper, are available online free of charge at~\url{http://thesis.mikeholler.me}.