\section{Background}
% Background section placeholder (move everything below this to another file eventually).
The process of creating an index is either costly or time consuming for authors, depending on whether they choose to create the index themselves or pay someone to do it.
The act of creating an index is labor and organization intensive, often requiring a text to be read multiple times while keeping track of entries electronically or on index cards.

Since indexing a book is so expensive and time consuming, it makes sense to see if computers can be used to generate an index that is nearly as accurate as human indexers.
To automate the indexing process, software exists to replace the indexer's index cards with a more efficient computerized organization system.
However, there is a new, rising interest in seeing if computers can generate indexes deterministically, without the help of human beings.
To do this, software engineers and researchers must apply natural language processing techniques in new and interesting ways.

\subsection{Indexing Process}
% How humans create indexes today.


Although statistics on the number of books that have indexes or the number new textbooks per year do not appear to exist, it seems a safe assumption to 

\subsection{Natural Language Processing}