\section{Background}
% Background section placeholder (move everything below this to another file eventually).
The process of creating an index is either costly or time consuming for authors, depending on whether they choose to create the index themselves or pay someone to do it.
The act of creating an index is labor and organization intensive, often requiring a text to be read multiple times while keeping track of entries electronically or on index cards.

Since indexing a book is so expensive and time consuming, it makes sense to see if computers can be used to generate an index that is nearly as accurate as human indexing professionals.
To automate the indexing process, software exists to replace the indexer's index cards with a more efficient computerized organization system.
However, there is a new, rising interest in seeing if computers can generate indexes deterministically, without the help of human beings.
To do this, software engineers and researchers must apply natural language processing techniques in new and interesting ways.

\subsection{Accuracy of Human and Automatic Indexes}

When attempting to automate the indexing process, it is important to understand what standards the software will be judged by.
In {\it Natural-Language Processing and Automatic Indexing}, an article published in {\it The Indexer} by C. Korycinski and Alan F. Newell, the authors cite relevant metrics for automatic indexing excerpted from Cleverdon's work:

\begin{quote}
\begin{enumerate}
\item if two people or groups of people construct thesauri in a given subject area, only 60\% of the index terms may be common to both;
\item if two experienced indexers index a given document using a given thesaurus, only 30\% of the index terms may be common to the two sets;
\item if two intermediaries [person who undertakes a database search on behalf of the user] search the same question on the same database on the same host, only 40\% of the output may be common to both searches;
\item if two scientists or engineers are asked to judge the relevance of a given set of documents to a given question, the area of agreement may not exceed 60\%.~\cite{automatic-indexing}
\end{enumerate}
\end{quote}

Of course, 2 is most relevant to this research, but the other three reveal how imprecisely human beings perform tasks that replicate another human's work.
In the statistics given above, indexing is the least precise of all of the human classification tasks mentioned above.
This makes the gold accuracy standard for an automatic indexer 30\%.
That is, a computer index compared to a human index should display at least the same proportion of similarity as a human generated index compared to another human generated index.
Now that a metric for index quality has been established, the next section discusses methods and strategies that can be used to create an automatic indexer that might be able to match this benchmark.

\subsection{Natural Language Processing}

Natural language processing (NLP) is, ``a form of computational linguistics in which natural-language texts are processed by computer (for automatic machine translation, literary text analysis, etc.)''~\cite{oed-nlp}.
Essentially, it is the area of study in computer science where language and logic meet.
Natural language processing (NLP) has existed since the beginning of electronic computers themselves (around 1940), but it has not existed in the modern sense until more recently.
Today, there is a growing focus on teaching computers to understand and learn the desired qualities of a text so they can be analyzed and extracted from large amounts of textual data~\cite{jurafsky}.

Internet search engines like Google rely on NLP techniques to generate their search results\footnote{As of March 1, 2014, Google has published 219 whitepapers on NLP topics, primarily on how to use NLP to improve search results~\cite{google-nlp}.}, text editing software uses NLP to detect grammatical errors in sentences~\cite{norvig}, and mobile applications use NLP to extract summaries from long form text~\cite{bit-of-news}.

\subsubsection{Machine Learning}

The process of teaching a computer to learn and understand information is called machine learning.
Machine learning sounds complicated, but ultimately it is about programming computers to answer questions about new, previously unseen data by correlating it to data it has previously seen (the data it has been {\it trained} on).
In {\it Natural Language Processing with Python}, an introductory NLP textbook, the authors use the example of name genders ({\it e.g.}, {\it Mike} is male and {\it Valerie} is female) to help readers understand how machine learning is used to help computers understand language~\cite{nlpwp}.
Given a list of names, a computer can be taught to determine (with reasonably high accuracy) whether a new, unseen name is masculine or feminine.
English readers use certain heuristics (shortcuts) to aid in determining the gender of a name, and computers can do the same.
For example, an average person might know that names ending in {\it -a} are typically feminine, while names ending in {\it -o} are typically masculine, so he or she understands that the last letter is a strong determinant of a name's gender, or {\it label}.
In NLP, these determinants are known as {\it features}.
With enough data about names and their genders (a {\it training set}), the probability of a letter determining a particular gender can be calculated, and a person could be {\it trained} using this data and the "last letter" feature to guess name genders with reasonable accuracy.
Of course, the last letter is not the only feature of a name that determines its gender, and by training with additional relevant features the classifier's guesses might be made more accurate.
Indeed, this process of determining the category (or {\it class}) of an item is called {\it classification}.
The above is an example a {\it supervised} machine learning, since training data is involved.
Unsupervised machine learning algorithms are outside the scope of this research, and will not be mentioned further in this paper.

\subsubsection{Document Classification}
% Document classification and automatic indexing
Document classification is a type of NLP that uses machine learning methods like those above, and is used in this research to create a computer generated index.
Document classification involves taking a piece of text (known as a {\it document}) as input and producing a label or {\it class} as output~\cite{jurafsky}.
Determining a name's gender can be thought of as document classification on a very small scale; a ``document'' in that case is a single name.
When applying document classification to automatic indexing, the size of the input documents must be defined.
Since classifiers label every input document with the most probable class\footnote{Many classifiers---including the one used in this research---do not have an ``unsure'' or ``ambiguous'' category; if the gender of a particular name is confusing, it is still assigned a gender according to the gender that is most likely, even if it is only slightly more likely than another gender.} the size of the document should be the smallest piece of text likely to contain an index reference.
To this author's knowledge, no research recommending a document size for automatic indexing tools exists; given the lack of research available, this study elects to use paragraphs as documents, such that a book containing $n$ paragraphs would be split up into $n$ unique documents.

Using a paragraph as the document size creates room for error, since one paragraph may not contain any index references, and another might contain more than one.
This is a problem, because a document classifier can only assign each document one and only one class.
While typically a paragraph will likely only have one index reference, the uncommon but possible cases of $r = 0$ and $r > 1$, where $r$ represents the number of references in a paragraph, must be handled gracefully and consistently.
The parameters for success in each of these cases are defined below:

\begin{itemize}
\item In the trivial $r = 1$ case, a document is successfully labeled when it is labeled with the same class the human generated index assigned to it.
\item In the case of $r > 1$, a document is successfully labeled when it is labeled with {\it one} of the same index labels as the human generated index assigned to it.
\item In the case of $r = 0$, a document can never be successfully labeled, since the classifier must assign the document a class. While this limitation is disappointing, it is impossible to avoid when using document classifiers. This case does not occur in this research, because the test set is specifically selected to only include documents that are referenced by one or more index entry.
\end{itemize}

This research makes the overarching assumption that a book with $n$ paragraphs has $n$ index references, or page number references in the book's index.
Clearly, this is an unrealistic assumption, but when dealing with document classification it is an assumption that must be made.

To clarify terminology, the following index is considered to have two index entries or labels, and five index or page number references:

\begin{center}
\begin{tabular}{l}
\textit{aardvark}, 12, 32, 54, 67 \\
\textit{apple}, 34
\end{tabular} 


\end{center}

\noindent This does {\it not} mean that a book is assumed to have $n$ unique words or phrases in its index, just that the collection of index words and phrases {\it points to} $n$ pages, since a singular index word might be referenced in multiple places (as seen in {\it aardvark} above).

\subsubsection{Na{\"i}ve Bayes Classifiers}

There are many different methods of classification, including decision trees, na{\"i}ve Bayes, and maximum entropy classifiers~\cite{nlpwp}, but this research only evaluates efficacy of the na{\"i}ve Bayes classifier in the task of automated indexing.
As far as classifiers go, \naive Bayes is one of the simplest implementations~\cite{rish}.
How it functions is described below:
\begin{quote}
The \naive Bayes classifier begins by calculating the prior probability of each label, which is determined by checking the frequency of each label in the training set. The contribution from each feature is then combined with this prior probability, to arrive at a likelihood estimate for each label. The label whose likelihood estimate is the highest is then associated to the input value.~\cite{nlpwp}
\end{quote}
\noindent Once trained, each feature is assigned one label.
In classification, any occurrence of a feature will ``push'' the document further towards the feature's assigned label.
When a feature ``pushes'' a document towards another label, it also pushes it towards similar labels, and an equal-but-opposite reaction hurts labels that are dissimilar from the occurring feature's label.

The decision to use \naive Bayes primarily motivated by simplicity, but it is important to acknowledge that this simplicity comes with a small cost in accuracy.
Since little research has been done on the use of document classification for automatic index generation, it made sense to begin with a simple implementation of such a classifier before even considering to work with something more complex like the maximum entropy classifier.
\Naive Bayes is considered simple because it makes a strong assumption that all features are independent from one another, that the appearance of one feature does not affect the possibility that another will appear.
This assumption is unrealistic as the probability of one feature might increase or decrease if another feature is present, creating an instance of conditional probability.
Ignoring conditional probability can lead to a phenomena called double counting.
This phenomena can be clearly seen if two features ``contains(apple)'' and ``contains(gala apple)'' are present in the same feature set.
The probability of ``apple'' occurring if `gala apple'' has already occurred is 1, and conversely the probability of ``gala apple'' occurring is increased when ``apple'' is present.
Although \naive Bayes' independence assumption does have some bearing on the result of the classification, research shows that this affect is largely negligible for most feature and data sets.

Irene Rish affirms in an IBM white paper that \naive Bayes works well with completely independent features {\it and} functionally dependent features (which should be surprising).
Her research also notes that, ``a better predictor of \naive Bayes accuracy is the amount of information about the class that is lost because of the independence assumption''~\cite{rish}.
These findings suggest that \naive Bayes should prove to be a sufficiently accurate predictor of the efficacy of using document classification for automatic indexing, despite the assumptions the algorithm makes about independence.

After deciding upon \naive Bayes, the next step for this research was to select an implementation of the algorithm to use for analysis.

\subsubsection{Natural Language Toolkit}
\label{sec:nltk}

This research makes use of Python's Natural Language Processing Toolkit (NLTK) library~\cite{nltk} for its implementation of \naive Bayes and other statistical data structures (e.g., frequency distributions).
Although many such libraries of machine learning and NLP tools exist~\cite{stanford-nlp,open-nlp}, NLTK was chosen for its popularity, flexibility, large selection of both NLP and machine learning tools, thorough documentation, and ease of use.

Publicly released in 2009, NLTK has become a staple in the world of NLP.
Even with well over 4,000 Python packages in the Python Package Index (PyPI), PyPI Ranking cites NLTK as having over 180,000 downloads, making it the \nth{273} most downloaded Python package of all time, and Python's most popular NLTK package~\cite{pypi-ranking}.
The latest available stable version of NLTK (v2.0.4) was used for the entirety of this research.

Python, the language NLTK is written for, is used for data analysis throughout this research.
Python is a mature and robust high-level scripting and prototyping language, and is therefore well-suited to the task of exploratory research.
This research uses the latest version of Python 2, since NLTK only provides alpha-level support for Python 3~\cite{nltk-install}.

\subsection{Wikipedia and Indexing}

Wikipedia is one of the largest centralized sources of informative non-fiction writing available today, and at the time of this writing currently has nearly 4.5 million articles in the English language alone~\cite{wikipedia}.
As a free, open resource, Wikipedia is a good candidate for use in Natural Language Processing because of the vast quantities of text it serves on a wide variety of topics.

When using a document classifier to create an index, the classifier needs to pull from a large pool of potential index labels to classify each piece of text with an appropriate index entry.
In order to classify a document as belonging to a particular class (or index label), the classifier needs to know what other text with that same class looks like (the training set).
A large pool of index labels must be supported by a proportionately large pool of training data.

If the title of a Wikipedia label is thought of as an index label, Wikipedia would seem to make for a vast, pre-labeled training set.
Using Wikipedia article titles as if they were index labels is intuitive, since both are short, one- to three-word noun phrases.
Wikipedia is used as the training set in this research because it represents a very large and comprehensive source of writing in an academic, textbook-like style, and its article titles are nearly identical in form to textbook index labels.
In theory, if Wikipedia can successfully aid in indexing a book from one subject, it will be easy to create an index from a book with a different subject while using a similar training set.